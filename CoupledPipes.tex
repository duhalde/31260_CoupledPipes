%%%%%%%%%%%%%%%%    Alle dokumentets opsætninger
\documentclass[a4paper,11pt,titlepage]{article}
\usepackage[danish]{babel}          % Brug af danske ord
\usepackage[utf8x]{inputenc}		% æ,ø,å [Latin1] under UNIX.
\usepackage[T1]{fontenc}            % Bruger en rigtig font til dansk output.
\usepackage{amsmath}                % Hvis man vil lave nogle frække matematikting
\usepackage{amssymb}                %  ---"---
\usepackage{mathabx}				% indeholder bl.a. \convolution
\usepackage{MnSymbol}
\usepackage[pdftex]{color,graphicx} % For at indsætte billeder
% \usepackage{pdfsync}
%\lstset{captionpos=b,float=*}
\usepackage{subfigure}  % lave figurer side om side
\usepackage{fullpage} % fikser siden så man har små sidemarginer.
\usepackage[version=3]{mhchem}
\usepackage[plainpages=false]{hyperref}
\hypersetup{colorlinks=true, linkcolor=blue, citecolor=red, urlcolor=black}
\usepackage[numbers, square]{natbib} % kan bruges til "Jones et al. (1990)" referencer
\bibpunct{[}{]}{,}{n}{,}{,~} % sørger for at \cite[rong,yuc] giver [1,2]
\usepackage{url}
\usepackage[isbn,issn,url]{dk-bib}
\usepackage{marginnote}

% \parindent=0pt % sletter den åndssvage indentering ved linieskift.
\newcommand{\eq}[1]{\begin{align*}#1\end{align*}}
\newcommand{\vnorm}[1]{\left|\!\left|#1\right|\!\right|}

%% Define a new 'leo' style for the package that will use a smaller font.
\makeatletter
\def\url@leostyle{%
  \@ifundefined{selectfont}{\def\UrlFont{\sf}}{\def\UrlFont{\small\ttfamily}}}
\makeatother
%% Now actually use the newly defined style.
\urlstyle{leo}

\graphicspath{pics/}

% Start af dokumentet

\begin{document}	

\begin{titlepage}
\centering \parindent=0pt
\newcommand{\HRule}{\rule{\textwidth}{1mm}}
\vspace*{\stretch{1}} \HRule\\[1cm]\Huge\bfseries
31260 - Advanced Acoustics\\Exercise Report\\[0.7cm]
\large Coupled Pipes\\[1cm]
\HRule\\[4cm]  \large by David Duhalde Rahbæk, s062050\\
Oliver Ackermann Lylloff, s082312\\
Mathias Immanuel Nielsen, s072101\\
\vspace*{\stretch{2}} \normalsize %
\begin{flushleft}
Technical University of Denmark\\
Department of Acoustic Technology\\
Teacher: Finn Jacobsen\\
\today \end{flushleft}
\end{titlepage}

\newpage

\tableofcontents
\newpage
\section{Introduction} % (fold)
\label{sec:introduction}

% section introduction (end)
\section{Theory} % (fold)
\label{sec:theory}

% section theory (end)
\section{Exercise}
\label{sec:exercise}

Description of exercise and results.

\newpage 
\section{Conclusion}
\label{sec:conclusion}

A discussion of the results...
% \newpage
% % \bibliographystyle{is-unsrt} % prøv også: is-unsrt, alpha, plain
% \bibliography{skabelon-bib}
% %\nocite{*} % \nocite bruges til bøger som skal med i litteraturlisten
%             % men som ikke er refereret til. Altså baggrundsstof mv.
\addcontentsline{toc}{section}{Litteratur}\label{cha:bib}
\bibliographystyle{unsrtnat-dk}
\bibliography{bib/New_BscBibliography.bib}

%\citationstyle{agsm}
%\nocite{*} %

% 
% \newpage
% \section{Litteraturliste}
% 
% %\bibliographystyle{unsrt}
% \begingroup
% \hypersetup{linkcolor=black}
% \printbibliography
% \endgroup

\end{document}
